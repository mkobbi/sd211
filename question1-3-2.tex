%% LyX 2.1.2 created this file.  For more info, see http://www.lyx.org/.
%% Do not edit unless you really know what you are doing.
\documentclass[english]{article}
\usepackage[T1]{fontenc}
\usepackage[latin9]{inputenc}
\usepackage{babel}
\begin{document}
Pour $Q\in\mathcal{M_{\left|\mathrm{U}\right|\times\left|\mathrm{C}\right|}},$$P\in\mathcal{M_{\left|\mathrm{C}\right|\times\left|\mathrm{I}\right|}}$,
on a :
\[
\overrightarrow{grad}f\left(P,Q\right)=\left(\begin{array}{c}
\frac{\partial f(P,Q)}{\partial P}\\
\frac{\partial f(P,Q)}{\partial Q}
\end{array}\right)
\]


et

\[
f\left(P,Q\right)=\frac{1}{2}\left\Vert 1_{K}\circ\left(R-QP\right)\right\Vert _{F}^{2}+\frac{\rho}{2}\left(\left\Vert Q\right\Vert _{F}^{2}+\left\Vert P\right\Vert _{F}^{2}\right)
\]


On pose

\[
\delta:(P,Q)\mapsto\left\Vert 1_{K}\circ\left(R-QP\right)\right\Vert _{F}^{2}
\]


et 
\[
M=1_{K}\circ\left(R-QP\right)
\]


On obtient :

\[
\begin{array}{ccc}
\partial M & = & \partial\left(1_{K}\circ\left(R-QP\right)\right)\\
 & = & \left(\partial1_{K}\right)\circ\left(R-QP\right)+1_{K}\circ\partial\left(R-QP\right)\\
 & = & 1_{K}\circ\left(\partial R-\partial\left(QP\right)\right)\\
 & = & -1_{K}\circ\left(\left(\partial Q\right)P+Q\left(\partial P\right)\right)
\end{array}
\]


D'o� 

\[
\begin{array}{ccc}
\frac{\partial f(P,Q)}{\partial P} & = & 2\left\langle M,\frac{\partial M}{\partial P}\right\rangle +\rho P\\
 & = & -1_{K}\circ Q^{t}\times1_{K}\circ\left(R-QP\right)+\rho P\\
\frac{\partial f(P,Q)}{\partial Q} & = & -1_{K}\circ\left(R-QP\right)\times1_{K}\circ P^{t}+\rho Q
\end{array}
\]


Les diff�rentielles du gradient ne sont pas born�es. Donc il n'est
pas Lipschitzien.
\end{document}
